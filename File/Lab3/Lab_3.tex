\refstepcounter{section}
\addlabcontentsline{Assignment \thesection: String}{05-02-2026}{\thepage}
\section*{\centering Assignment \thesection: String}
%-----------------------------------------------
% Task 1
% -----------------------------------------------
\subsection{Write a program to Count Vowels in a String}
\begin{lstlisting}[style=pythonStyle, caption={}]
package Strings;
// Count vowels name in String
public class CountVowels {
    public static void main(String[] args) {
        String str = "Eleven";
        int count = 0;
        for(int i=0; i<str.length();i++){
            char ch = str.charAt(i);
            if("AEIOUaeiou".indexOf(ch) != -1){
                count++;
            }
        }
        System.out.println("Vowels: "+ count);
    }
}
\end{lstlisting}
\subsubsection*{Output}
% \vspace{-1em}
\begin{figure}[H]
    \centering
    \vspace{-1em}
    \includegraphics[width=1.0\textwidth]{Lab3/Images/count_vowels.png}
    \caption{Output of Count Vowels Program}
    \label{fig:count_vowels_output}
\end{figure}
%-----------------------------------------------
% Task 2
% -----------------------------------------------
\subsection{Palindrome String}
\begin{lstlisting}[style=pythonStyle, caption={}]
package Strings;
// Word is spelled in same way forward or backward. Eg: NITIN, MADAM
//Paldindrome
public class Palindrome {
    public static void main(String[] args) {
        String str = "nitin";
        String rev = "";

        for(int i=str.length()-1; i>=0;i--){
            rev += str.charAt(i);
        }

        if (str.equals(rev)) {
            System.out.println(str + " is a palindrome.");
        }else{
            System.out.println(str + " is not a palindrome.");
        }
    }
}
\end{lstlisting}
\subsubsection*{Output}
% \vspace{-1em}
\begin{figure}[H]
    \centering
    \vspace{-1em}
    \includegraphics[width=1.0\textwidth]{Lab3/Images/palindrome.png}
    \caption{Output of Palindrome Program}
    \label{fig:palindrome_program_output}
\end{figure}
%-----------------------------------------------    
% Task 3
% -----------------------------------------------
\subsection{Remove Spaces from a String}
\begin{lstlisting}[style=pythonStyle, caption={}]
package Strings;
// Remove spaces from string
public class RemoveSpaces {
    public static void main(String[] args) {
        String str = "Every Expert Was Once A Beginner.";
        System.out.println(str.replace(" ", ""));
    }
}

\end{lstlisting}
\subsubsection*{Output}
% \vspace{-1em}
\begin{figure}[H]
    \centering
    \vspace{-1em}
    \includegraphics[width=1.0\textwidth]{Lab3/Images/remove_space.png}
    \caption{Output of Remove Spaces Program}
    \label{fig:remove_spaces_output}
\end{figure}
%-----------------------------------------------
% Task 4
% -----------------------------------------------
\subsection{Reverse a String}
\begin{lstlisting}[style=pythonStyle, caption={}]
package Strings;

public class ReverseString {
    
    public static void main(String[] args) {
        
        String str = "Hello";
        String rev = "";

        for(int i = str.length()-1; i>=0; i--){
            rev = rev + str.charAt(i);
        }
        System.out.println("Reversed String: "+ rev);

    }
}
\end{lstlisting}
\subsubsection*{Output}
% \vspace{-1em}
\begin{figure}[H]
    \centering
    \vspace{-1em}
    \includegraphics[width=1.0\textwidth]{Lab3/Images/reverse_string.png}
    \caption{Output of Reverse String Program}
    \label{fig:reverse_string_output}
\end{figure}

%-----------------------------------------------
