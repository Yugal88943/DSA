\documentclass[11pt]{hcmut-report}

% ---------- BASIC PACKAGES ----------
\usepackage[utf8]{inputenc}          % UTF-8 encoding (keep if using pdflatex)
\usepackage{graphicx}                % For images
\usepackage{float}                   % For [H] placement
\usepackage[table]{xcolor}           % Table colors
\usepackage{booktabs, longtable, makecell, array, multicol, multirow, siunitx, tabularx}
\usepackage{gensymb, textcomp}       % Symbols like degree, trademark, etc.

% ---------- CAPTIONS AND SUBCAPTIONS ----------
\usepackage{caption}
\usepackage{subcaption}

% ---------- CODE + OUTPUT STYLING ----------
\usepackage{listings}
\usepackage{xcolor}
\usepackage{tcolorbox}
\tcbuselibrary{listings, skins, breakable}
\usepackage{amssymb}

% Python code style
\lstdefinestyle{pythonStyle}{
    language=Python,
    basicstyle=\ttfamily\small,
    keywordstyle=\color{blue}\bfseries,
    stringstyle=\color{teal},
    commentstyle=\color{gray},
    showstringspaces=false,
    breaklines=true,
    frame=single,
    numbers=left,
    numberstyle=\tiny,
    tabsize=4
}

% Output box
\newtcolorbox{outputbox}{
  enhanced,
  breakable,
  colback=gray!5!white,
  colframe=black!70!white,
  fonttitle=\bfseries,
  title=Output,
  boxrule=0.4pt,
  left=4pt,
  right=4pt,
  top=2pt,
  bottom=2pt,
  listing only,
  width=\textwidth,
  listing options={
    basicstyle=\ttfamily\small,
    breaklines=true,
    showstringspaces=false,
    numbers=none,
    frame=none,
    aboveskip=4pt,    % space above
    belowskip=4pt,    % space below
    columns=fixed,
    keepspaces=true
  }
}

\usepackage{tcolorbox}
\tcbuselibrary{breakable, skins}

% New box for Python output only
\newtcolorbox{Numbox}{
  enhanced,
  breakable,
  colback=gray!5!white,
  colframe=black!70!white,
  fonttitle=\bfseries,
  title=Python Output,
  boxrule=0.4pt,
  left=4pt,
  right=4pt,
  top=2pt,
  bottom=2pt,
  sharp corners,
  boxsep=4pt,
  before skip=6pt,
  after skip=6pt,
  enhanced jigsaw,
  frame hidden,
}



% ---------- DOCUMENT FORMATTING ----------
\usepackage{setspace}
\onehalfspacing
\usepackage{enumitem}                % Better enumerate/itemize
\usepackage{tocloft}                 % Custom ToC control
\usepackage{xparse}                  % For custom command definitions
\usepackage{fancyvrb}                % For better verbatim

% ---------- CUSTOM COMMAND ----------
\NewDocumentCommand{\addlabcontentsline}{mmm}{%
  \addtocontents{toc}{%
    \noindent\textbf{#1}%
    \hfill \textbf{#2}%
    \hfill \textbf{#3}\par
  }%
}

% ---------- INDEX ----------
\usepackage{makeidx}
\makeindex

% ---------- DOCUMENT BEGINS ----------
\begin{document}

\pagenumbering{roman}
\thispagestyle{empty}
\graphicspath{{Figures/PNG/}{Figures/}}
\begin{center}
        \Large \textit{Laboratory Assignment File\\}
	\vspace{1mm}
        \large \textit{for\\}
	\vspace{5mm}
	\Huge \textcolor{black}{\textbf{Data Structure and Algorithms }} \\
	\vspace{10mm}
	% \large \textit{Submitted in partial fulfillment of the requirements \\ for the award of the degree of} \\

	\Large {\textbf{Master of Technology}} \\

    \vspace{1mm}
    \Large {in} \\
    \vspace{1mm}
    \large \textbf{Computer Science \& Engineering} \\
    \vspace{5mm}
    \large \textbf{Submitted by} \\
    \vspace{5mm}
	\textbf{{\Large Yugal} \\ {\large (Roll no. 25903053)}} \\
	\vspace{8mm}
	
	\begin{figure}[h!]
		\centering
		\includegraphics[height=3.5cm,width=3.5cm]{logo_250_1 (1).png}
	\end{figure}
	\par\vspace{5mm}
	\large \textbf{Department of Computer Science \& Engineering\\
			Dr. B R Ambedkar National Institute of Technology Jalandhar
			 \\Punjab, India-144008 \\
			May, 2026}
\end{center}
\thispagestyle{empty}


\tableofcontents
\clearpage
\listoffigures
\clearpage
% \listoftables
\clearpage

\pagenumbering{arabic}
\setcounter{page}{1}

% ---------- LAB INCLUSIONS ----------

% Assignment 1: Basic Programs in Java
%-----------------------------------------------
\refstepcounter{section}
\addlabcontentsline{Assignment \thesection: Basic Programs }{28-01-2026}{\thepage}
\section*{\centering Assignment \thesection: Basic Programs}

%-----------------------------------------------
% Task 1
% -----------------------------------------------
\subsection{Hello World}
\begin{lstlisting}[style=pythonStyle, caption={}]
public class Hello{
    public static void main(String arg[]){
    System.out.println("Hello");
    }
}
\end{lstlisting}

%-----------------------------------------------
% Task 2
% -----------------------------------------------
\subsection{Write a program to show Array}
\begin{lstlisting}[style=pythonStyle, caption={}]
// Store and display the marks
public class Array{
    public static void main(String... args){
        int[] marks = {70,75,80,90,95};
        for(int i=0;i<marks.length;i++){
            // System.out.println("Student Marks: "+marks[i]);  
            System.out.println("Student " + (i+1) +":"+marks[i]);
        }
    }
    
}

\end{lstlisting}
\subsubsection*{Output}
% \vspace{-1em}
\begin{figure}[H]
    \centering
    \vspace{-1em}
    \includegraphics[width=1.0\textwidth]{Lab1/Images/array.png}
    \caption{Program to show Array}
    \label{fig:array_output}
\end{figure}

%-----------------------------------------------
% Task 3
% -----------------------------------------------
\subsection{Write a program to show sum and average}
\begin{lstlisting}[style=pythonStyle, caption={}]
// Sum and average of array elements
class Sum_Average {
    
    public static void main(String[] args){
        int[] a = {20,30,40,50,60};
        int sum = 0;
        
        for(int x:a){
            sum += x;
        }
        double average = (double) sum/a.length;
        
        System.out.println("Sum: "+ sum);
        System.out.println("Average: "+ average);
    }
}

\end{lstlisting}
\subsubsection*{Output}
% \vspace{-1em}
\begin{figure}[H]
    \centering
    \vspace{-1em}
    \includegraphics[width=1.0\textwidth]{Lab1/Images/sum_average.png}
    \caption{Program to show sum and average}
    \label{fig:sum_average_output}
\end{figure}    




\refstepcounter{section}
\addlabcontentsline{Assignment \thesection: Sorting Algorithms}{02-02-2026}{\thepage}
\section*{\centering Assignment \thesection: Sorting Algorithms}
%-----------------------------------------------
% Task 1
% -----------------------------------------------
\subsection{Write a program to show Bubble Sort}
\begin{lstlisting}[style=pythonStyle, caption={}]
// Sorting of array in Bubble Sort
public class BubbleSort{
    public static void main(String[] args){
        int arr[] = {5,1,4,2,8};
        for(int i=0; i<arr.length-1; i++){
            for(int j=0; j<arr.length-i-1; j++){
                if(arr[j]>arr[j+1]){
                    int temp = arr[j];
                    arr[j] = arr[j+1];
                    arr[j+1] = temp;
                }
            }
        }
        for(int num : arr){
            System.out.print(num + " ");
        }
    }
}
\end{lstlisting}
\subsubsection*{Output}
% \vspace{-1em}
\begin{figure}[H]
    \centering
    \vspace{-1em}
    \includegraphics[width=1.0\textwidth]{Lab2/Images/bubble_sort.png}
    \caption{Program to show Bubble Sort}
    \label{fig:bubble_sort_output}
\end{figure}
%-----------------------------------------------
% Task 2
% -----------------------------------------------
\subsection{Write a program to show Insertion Sort}
\begin{lstlisting}[style=pythonStyle, caption={}]
// Insertion Sort
public class InsertionSort {
    public static void main(String[] args) {
        int arr[] = {1,4,5,2,3,7,9,8,0,6};
        for(int i = 1;i<arr.length; i++){
            int key = arr[i];
            int j = i-1;
            while (j>=0 && arr[j]>key) {
                arr[j+1] = arr[j];
                j--;
            }
            arr[j+1] = key;
        }
        for(int num : arr){
            System.out.print(num + " ");
        }
    }    
}
\end{lstlisting}
\subsubsection*{Output}
% \vspace{-1em}
\begin{figure}[H]
    \centering
    \vspace{-1em}
    \includegraphics[width=1.0\textwidth]{Lab2/Images/insertion_sort.png}
    \caption{Program to show Insertion Sort}
    \label{fig:insertion_sort_output}
\end{figure}
%-----------------------------------------------    
% Task 3
% -----------------------------------------------
\subsection{Write a program to show Selection Sort}
\begin{lstlisting}[style=pythonStyle, caption={}]
    //Selection Sort 
public class SelectionSort {
    public static void main(String[] args) {
        int arr[] = {3,5,2,6,-1,0,4,7,9,8,-2};

        for(int i=0;i<arr.length-1;i++){
            int minIndex = i;
            for(int j = i+1; j<arr.length;j++){
                if(arr[j]<arr[minIndex]){
                    minIndex = j;

                }
            }
            int temp = arr[minIndex];
            arr[minIndex] = arr[i];
            arr[i] = temp;
        }
        for(int num : arr){
            System.out.print(num + " ");

        }
    }
    
}
\end{lstlisting}
\subsubsection*{Output}
% \vspace{-1em}
\begin{figure}[H]
    \centering
    \vspace{-1em}
    \includegraphics[width=1.0\textwidth]{Lab2/Images/selection_sort.png}
    \caption{Program to show Selection Sort}
    \label{fig:selection_sort_output}
\end{figure}
%-----------------------------------------------
% Task 4
% -----------------------------------------------
\subsection{Write a program to show Merge Sort}
\begin{lstlisting}[style=pythonStyle, caption={}]
    //Merge Sort : Divide and Conquer
import java.util.Arrays;

public class MergeSort {
   
    // Main merge sort function
    public static void mergeSort(int[] arr, int left, int right){
        if(left < right){
            int mid =  left + (right-left)/2;
            // Sort Left Half
            mergeSort(arr, left, mid);
            // Sort Right Half
            mergeSort(arr, mid+1, right);
            // Merge Both Halves
            merge(arr, left, mid, right); 
        }
    }

    // Merge two sorted subarrays
    public static void merge(int[] arr, int left, int mid, int right){
        int n1 = mid-left+1;
        int n2 = right-mid;
        int[] l = new int[n1];
        int[] r = new int[n2];

        // Copy data to temp arrays
        for(int i=0; i<n1; i++)
            l[i] = arr[left+i];
        for(int j=0; j<n2; j++)
            r[j] = arr[mid+1+j];

        int i = 0, j=0, k=left;

        // Merge temp arrays back into array/arr
        while (i<n1 && j<n2) {
            if (l[i]<=r[j]) {
                arr[k++] = l[i++];
            }
            else{
                arr[k++] = r[j++];
            }
        }

        // Copy remaining elements
        while (i<n1) {
            arr[k++] = l[i++];
        }
        while (j<n2) {
            arr[k++] = r[j++];
        }
    }

    // Driver Code
    public static void main(String[] args) {
        
        int[] arr = {38,27,43,3,9,82,10};
        System.out.println("Before Sorting: ");
        System.out.println(Arrays.toString(arr));

        mergeSort(arr, 0, arr.length-1);
        
        System.out.println("After Sorting: ");
        System.out.println(Arrays.toString(arr));

    }
}
\end{lstlisting}
\subsubsection*{Output}
% \vspace{-1em}
\begin{figure}[H]
    \centering
    \vspace{-1em}
    \includegraphics[width=1.0\textwidth]{Lab2/Images/merge_sort.png}
    \caption{Program to show Merge Sort}
    \label{fig:merge_sort_output}
\end{figure}
%-----------------------------------------------
% Task 5
% -----------------------------------------------
\subsection{Write a program to show Quick Sort}
\begin{lstlisting}[style=pythonStyle, caption={}]
    // Quick Sort : small -- pivot -- larger
import java.util.Arrays;

public class QuickSort {

    // Main quick sort function
    public static void quickSort(int[] arr, int low, int high){
        if (low<high) {
            int pivotIndex = partition(arr, low, high);
            
            // Sort elements before and after partition
            quickSort(arr, low, pivotIndex-1);
            quickSort(arr, pivotIndex+1, high);
        }
    }

    // Partition Function 
    public static int partition(int[] arr, int low, int high){
        int pivot = arr[high]; // Choose last element as pivot
        int i = low-1;
        for(int j=low; j<high; j++){
            if (arr[j]<pivot) {
                i++;

                // Swap arr[i] and arr[j]
                int temp = arr[i];
                arr[i] = arr[j];
                arr[j] = temp;
            }
        }

        // Place pivot at correct position
        int temp = arr[i+1];
        arr[i+1] = arr[high];
        arr[high] = temp;

        return i+1;
    }

    // Driver code
    public static void main(String[] args) {
        
        int[] arr = {34,2,56,4,89,65,1,44};
        System.out.println("Before Sorting: ");
        System.out.println(Arrays.toString(arr));

        quickSort(arr, 0, arr.length-1);
        System.out.println("After Sorting: ");
        System.out.println(Arrays.toString(arr));

    }
}
\end{lstlisting}
\subsubsection*{Output}
% \vspace{-1em}
\begin{figure}[H]
    \centering
    \vspace{-1em}
    \includegraphics[width=1.0\textwidth]{Lab2/Images/quick_sort.png}
    \caption{Program to show Quick Sort}
    \label{fig:quick_sort_output}
\end{figure}
%-----------------------------------------------

% \refstepcounter{section}
\addlabcontentsline{Assignment \thesection: String}{05-02-2026}{\thepage}
\section*{\centering Assignment \thesection: String}
%-----------------------------------------------
% Task 1
% -----------------------------------------------
\subsection{Write a program to Count Vowels in a String}
\begin{lstlisting}[style=pythonStyle, caption={}]
package Strings;
// Count vowels name in String
public class CountVowels {
    public static void main(String[] args) {
        String str = "Eleven";
        int count = 0;
        for(int i=0; i<str.length();i++){
            char ch = str.charAt(i);
            if("AEIOUaeiou".indexOf(ch) != -1){
                count++;
            }
        }
        System.out.println("Vowels: "+ count);
    }
}
\end{lstlisting}
\subsubsection*{Output}
% \vspace{-1em}
\begin{figure}[H]
    \centering
    \vspace{-1em}
    \includegraphics[width=1.0\textwidth]{Lab3/Images/count_vowels.png}
    \caption{Output of Count Vowels Program}
    \label{fig:count_vowels_output}
\end{figure}
%-----------------------------------------------
% Task 2
% -----------------------------------------------
\subsection{Palindrome String}
\begin{lstlisting}[style=pythonStyle, caption={}]
package Strings;
// Word is spelled in same way forward or backward. Eg: NITIN, MADAM
//Paldindrome
public class Palindrome {
    public static void main(String[] args) {
        String str = "nitin";
        String rev = "";

        for(int i=str.length()-1; i>=0;i--){
            rev += str.charAt(i);
        }

        if (str.equals(rev)) {
            System.out.println(str + " is a palindrome.");
        }else{
            System.out.println(str + " is not a palindrome.");
        }
    }
}
\end{lstlisting}
\subsubsection*{Output}
% \vspace{-1em}
\begin{figure}[H]
    \centering
    \vspace{-1em}
    \includegraphics[width=1.0\textwidth]{Lab3/Images/palindrome.png}
    \caption{Output of Palindrome Program}
    \label{fig:palindrome_program_output}
\end{figure}
%-----------------------------------------------    
% Task 3
% -----------------------------------------------
\subsection{Remove Spaces from a String}
\begin{lstlisting}[style=pythonStyle, caption={}]
package Strings;
// Remove spaces from string
public class RemoveSpaces {
    public static void main(String[] args) {
        String str = "Every Expert Was Once A Beginner.";
        System.out.println(str.replace(" ", ""));
    }
}

\end{lstlisting}
\subsubsection*{Output}
% \vspace{-1em}
\begin{figure}[H]
    \centering
    \vspace{-1em}
    \includegraphics[width=1.0\textwidth]{Lab3/Images/remove_space.png}
    \caption{Output of Remove Spaces Program}
    \label{fig:remove_spaces_output}
\end{figure}
%-----------------------------------------------
% Task 4
% -----------------------------------------------
\subsection{Reverse a String}
\begin{lstlisting}[style=pythonStyle, caption={}]
package Strings;

public class ReverseString {
    
    public static void main(String[] args) {
        
        String str = "Hello";
        String rev = "";

        for(int i = str.length()-1; i>=0; i--){
            rev = rev + str.charAt(i);
        }
        System.out.println("Reversed String: "+ rev);

    }
}
\end{lstlisting}
\subsubsection*{Output}
% \vspace{-1em}
\begin{figure}[H]
    \centering
    \vspace{-1em}
    \includegraphics[width=1.0\textwidth]{Lab3/Images/reverse_string.png}
    \caption{Output of Reverse String Program}
    \label{fig:reverse_string_output}
\end{figure}

%-----------------------------------------------

% \refstepcounter{section}
\addlabcontentsline{Assignment \thesection: Iteration}{07-02-2026}{\thepage}
\section*{\centering Assignment \thesection: Iteration}
%-----------------------------------------------
% Task 1
% -----------------------------------------------
\subsection{Factorial using Iteration}
\begin{lstlisting}[style=pythonStyle, caption={}]
package Iteration;
// Factorial program using iteration
public class FactorialUsingIteration {
    // Driver code
    public static void main(String[] args) {
        int n = 5;
        int fact = 1;
        for(int i=1; i<=n; i++){
            fact *= i;
        }
        System.out.println("Factorial : "+fact);
    }
}

\end{lstlisting}
\subsubsection*{Output}
% \vspace{-1em}
\begin{figure}[H]
    \centering
    \vspace{-1em}
    \includegraphics[width=1.0\textwidth]{Lab4/Images/fact_iteration.png}
    \caption{Output of Factorial using Iteration Program}
    \label{fig:fact_iteration_output}
\end{figure}


%-----------------------------------------------

% \include{Lab4_1/Lab_4_1}
% \refstepcounter{section}
\addlabcontentsline{Assignment \thesection: Searching}{09-02-2026}{\thepage}
\section*{\centering Assignment \thesection: Searching}
%-----------------------------------------------
% Task 1
% -----------------------------------------------
\subsection{Binary Search}
\begin{lstlisting}[style=pythonStyle, caption={}]
// Binary Search
package Searching;
import java.util.Scanner;

public class BinarySearch {
    
    //Driver Code
    public static void main(String[] args) {
        Scanner sc = new Scanner(System.in);
        int[] arr = new int[20];// OK
        int n = 0, choice, key;
        
        do{
            System.out.println("\n ---Binary Search---");
            System.out.println("1. Insert Elements");
            System.out.println("2. Display Elements");
            System.out.println("3. Binary Search");
            System.out.println("4. Exit");

            System.out.print("Enter your choice: ");
            choice = sc.nextInt();

            switch (choice) {
                case 1:
                    System.out.print("Enter number of elements: ");
                    n = sc.nextInt();
                    System.out.println("Enter Elements: ");

                    for(int i=0; i<n; i++){
                        arr[i] = sc.nextInt();
                    }
                    break;

                case 2:
                    System.out.print("Array Elements: ");
                    for(int i=0; i<n; i++){
                        System.out.print(arr[i] + " ");
                    }
                    System.out.println();
                    break;

                case 3:
                    System.out.print("Enter element to search: ");
                    key = sc.nextInt();
                    int low = 0, high = n-1;
                    boolean found = false;
                    while (low <= high) {
                        int mid = (low+high) / 2;
                        if (arr[mid] == key) {
                            System.out.println("Element found at index: " + mid);
                            found = true;
                            break;
                        }
                        else if (arr[mid] < key) {
                            low = mid+1;
                        }
                        else{
                            high = mid-1;
                        }
                    }

                    if (!found) 
                        System.out.println("Element not found");
                    break;

                case 4:
                    System.out.println("Exiting Program");
                    break;
                
                default:
                    System.out.println("Invalid Choice"); // break not compulsory
            }
        }
        while (choice != 4);
        sc.close();
    }
}

\end{lstlisting}
\subsubsection*{Output}
% \vspace{-1em}
\begin{figure}[H]
    \centering
    \vspace{-1em}
    \includegraphics[width=1.0\textwidth]{Lab5/Images/binary_search1.png}
    % \includegraphics[width=1.0\textwidth]{Lab5/Images/binary_search2.png}
    \caption{Output of Binary Search Program}
    \label{fig:binary_search_output}
\end{figure}
\begin{figure}[H]
    \centering
    \vspace{-1em}
    % \includegraphics[width=1.0\textwidth]{Lab5/Images/binary_search1.png}
    \includegraphics[width=1.0\textwidth]{Lab5/Images/binary_search2.png}
    \caption{Output of Binary Search Program}
    \label{fig:binary_search_output_2}
\end{figure}
%-----------------------------------------------
% Task 2
% -----------------------------------------------
\subsection{Linear Search}
\begin{lstlisting}[style=pythonStyle, caption={}]
package Searching;

import java.util.Scanner;

public class LinearSearch {
    public static void main(String[] args) {
        Scanner sc = new Scanner(System.in);
        int[] arr = new int[10]; 
        int n = 0, choice, key;
        boolean found;

        do{
            System.out.println("\n ---Linear Search Menu---");
            System.out.println("1. Insert Elements");
            System.out.println("2. Display Elements");
            System.out.println("3. Linear Search");
            System.out.println("4. Exit");

            System.out.print("Enter your choice: ");
            choice = sc.nextInt();

            switch (choice) {
                case 1:
                    System.out.print("Enter number of elements: ");
                    n = sc.nextInt();
                    System.out.println("Enter Elements: ");

                    for(int i=0; i<n; i++){
                        arr[i] = sc.nextInt();
                    }
                    break;
                
                case 2:
                    System.out.print("Array Elements: ");
                    for(int i=0; i<n; i++){
                        System.out.print(arr[i] + " ");
                    }
                    System.out.println();
                    break;

                case 3:
                    System.out.print("Enter element to search: ");
                    key = sc.nextInt();
                    found=false;

                    for(int i=0; i<n; i++){
                        if (arr[i] == key) {
                            System.out.println("Element found at index: " + i);
                            found = true;
                            break;
                        }
                    }

                    if (!found) 
                        System.out.println("Element not found");
                        break;
                    
                case 4:
                    System.out.println("Exiting Program");
                    break;

                default:
                    System.out.println("Invalid Choice");
                    break;
            }
        }
        
        while (choice != 4);
        sc.close();
    }
}

\end{lstlisting}
\subsubsection*{Output}
% \vspace{-1em}
\begin{figure}[H]
    \centering
    \vspace{-1em}
    \includegraphics[width=1.0\textwidth]{Lab5/Images/linear_search1.png}
    % \includegraphics[width=1.0\textwidth]{Lab5/Images/linear_search2.png}
    \caption{Output of Linear Search Program}
    \label{fig:linear_search_output}
\end{figure}
\begin{figure}[H]
    \centering
    \vspace{-1em}
    % \includegraphics[width=1.0\textwidth]{Lab5/Images/linear_search1.png}
    \includegraphics[width=1.0\textwidth]{Lab5/Images/linear_search2.png}
    \caption{Output of Linear Search Program}
    \label{fig:linear_search_output_2}
\end{figure}


%-----------------------------------------------

% \refstepcounter{section}
\addlabcontentsline{Assignment \thesection: Recursion}{11-02-2026}{\thepage}
\section*{\centering Assignment \thesection: Recursion}
%-----------------------------------------------
% Task 1
% -----------------------------------------------
\subsection{Factorial of a Number using Recursion}
\begin{lstlisting}[style=pythonStyle, caption={}]
package Recursion;

import java.util.Scanner;

// Factorial Program --> eg: 4! = 4*3*2*1 = 24
public class Factorial {

    static int factorial(int n){
        if (n == 0) 
            return 1;
        return n * factorial(n-1);
    }

    //Driver Code
    public static void main(String[] args) {

        System.out.println("Enter any number: ");
        Scanner sc = new Scanner(System.in);
        int num = sc.nextInt();
        System.out.println("Factorial of "+ num + " is "+ factorial(num));
        sc.close();
    }
}

\end{lstlisting}
\subsubsection*{Output}
% \vspace{-1em}
\begin{figure}[H]
    \centering
    \vspace{-1em}
    \includegraphics[width=1.0\textwidth]{Lab6/Images/factorial.png}
    \caption{Output of Factorial Program}
    \label{fig:factorial_output}
\end{figure}
%-----------------------------------------------
% Task 2
% -----------------------------------------------
\subsection{Fibonacci Series using Recursion}
\begin{lstlisting}[style=pythonStyle, caption={}]
package Recursion;
import java.util.Scanner;

//Fibonacci Series --> 0,1,0+1=1,1+1=2,2+1=3...
public class Fibonacci {
    static int fibonacci(int n){
        if (n == 0)
            return 0;
        if (n == 1)
            return 1;
        return fibonacci(n-1) + fibonacci(n-2);
    }
    //Driver Code
    public static void main(String[] args) {

        Scanner sc = new Scanner(System.in);
        System.out.println("Enter Element: ");
        
        int terms = sc.nextInt();
        for(int i = 0; i<terms; i++){
            System.out.print(fibonacci(i) + " ");
            sc.close();
        } 
    }
}

\end{lstlisting}
\subsubsection*{Output}
% \vspace{-1em}
\begin{figure}[H]
    \centering
    \vspace{-1em}
    \includegraphics[width=1.0\textwidth]{Lab6/Images/fibonacci.png}
    \caption{Output of Fibonacci Program}
    \label{fig:fibonacci_output}
\end{figure}
%-----------------------------------------------    
% Task 3
% -----------------------------------------------
\subsection{Palindrome String using Recursion}
\begin{lstlisting}[style=pythonStyle, caption={}]
package Strings;
// Remove spaces from string
public class RemoveSpaces {
    public static void main(String[] args) {
        String str = "Every Expert Was Once A Beginner.";
        System.out.println(str.replace(" ", ""));
    }
}

\end{lstlisting}
\subsubsection*{Output}
% \vspace{-1em}
\begin{figure}[H]
    \centering
    \vspace{-1em}
    \includegraphics[width=1.0\textwidth]{Lab6/Images/palindrome.png}
    \caption{Output of Palindrome Program}
    \label{fig:palindrome_output}
\end{figure}
%-----------------------------------------------
% Task 4
% -----------------------------------------------
\subsection{Power of a number using Recursion}
\begin{lstlisting}[style=pythonStyle, caption={}]
package Strings;

public class ReverseString {
    
    public static void main(String[] args) {
        
        String str = "Hello";
        String rev = "";

        for(int i = str.length()-1; i>=0; i--){
            rev = rev + str.charAt(i);
        }
        System.out.println("Reversed String: "+ rev);

    }
}
\end{lstlisting}
\subsubsection*{Output}
% \vspace{-1em}
\begin{figure}[H]
    \centering
    \vspace{-1em}
    \includegraphics[width=1.0\textwidth]{Lab6/Images/power_of_number.png}
    \caption{Output of Power of Number Program}
    \label{fig:power_of_number_output}
\end{figure}

%-----------------------------------------------
% Task 5
% -----------------------------------------------
\subsection{Print numbers using Recursion}
\begin{lstlisting}[style=pythonStyle, caption={}]
package Strings;

public class ReverseString {
    
    public static void main(String[] args) {
        
        String str = "Hello";
        String rev = "";

        for(int i = str.length()-1; i>=0; i--){
            rev = rev + str.charAt(i);
        }
        System.out.println("Reversed String: "+ rev);

    }
}
\end{lstlisting}
\subsubsection*{Output}
% \vspace{-1em}
\begin{figure}[H]
    \centering
    \vspace{-1em}
    \includegraphics[width=1.0\textwidth]{Lab6/Images/numbers_print.png}
    \caption{Output of Print Numbers Program}
    \label{fig:numbers_print_output}
\end{figure}

%-----------------------------------------------
% Task 6
% -----------------------------------------------
\subsection{Recursion Factorial of a Number}
\begin{lstlisting}[style=pythonStyle, caption={}]
package Strings;

public class ReverseString {
    
    public static void main(String[] args) {
        
        String str = "Hello";
        String rev = "";

        for(int i = str.length()-1; i>=0; i--){
            rev = rev + str.charAt(i);
        }
        System.out.println("Reversed String: "+ rev);

    }
}
\end{lstlisting}
\subsubsection*{Output}
% \vspace{-1em}
\begin{figure}[H]
    \centering
    \vspace{-1em}
    \includegraphics[width=1.0\textwidth]{Lab6/Images/recurse_factorial.png}
    \caption{Output of Recursion Factorial Program}
    \label{fig:recurse_factorial_output}
\end{figure}

%-----------------------------------------------
% Task 7
% -----------------------------------------------
\subsection{Reverse Number using Recursion}
\begin{lstlisting}[style=pythonStyle, caption={}]
package Strings;

public class ReverseString {
    
    public static void main(String[] args) {
        
        String str = "Hello";
        String rev = "";

        for(int i = str.length()-1; i>=0; i--){
            rev = rev + str.charAt(i);
        }
        System.out.println("Reversed String: "+ rev);

    }
}
\end{lstlisting}
\subsubsection*{Output}
% \vspace{-1em}
\begin{figure}[H]
    \centering
    \vspace{-1em}
    \includegraphics[width=1.0\textwidth]{Lab6/Images/reverse_number.png}
    \caption{Output of Reverse Number Program}
    \label{fig:reverse_number_output}
\end{figure}

%-----------------------------------------------
% Task 8
% -----------------------------------------------
\subsection{String Recursion Menu}
\begin{lstlisting}[style=pythonStyle, caption={}]
package Strings;

public class ReverseString {
    
    public static void main(String[] args) {
        
        String str = "Hello";
        String rev = "";

        for(int i = str.length()-1; i>=0; i--){
            rev = rev + str.charAt(i);
        }
        System.out.println("Reversed String: "+ rev);

    }
}
\end{lstlisting}
\subsubsection*{Output}
% \vspace{-1em}
\begin{figure}[H]
    \centering
    % \vspace{-1em}
    \includegraphics[width=0.8\textwidth]{Lab6/Images/menu1.png}
    % \includegraphics[width=1.0\textwidth]{Lab6/Images/menu2.png}
    \caption{Output of String Recursion Menu Program}
    \label{fig:string_recursion_menu_output}
\end{figure}
\begin{figure}[H]
    \centering
    % \vspace{-1em}
    % \includegraphics[width=1.0\textwidth]{Lab6/Images/menu1.png}
    \includegraphics[width=0.8\textwidth]{Lab6/Images/menu2.png}
    \caption{Output of String Recursion Menu Program}
    \label{fig:string_recursion_menu_output_2}
\end{figure}

%-----------------------------------------------
% Task 9
% -----------------------------------------------
\subsection{Sum of digits using Recursion}
\begin{lstlisting}[style=pythonStyle, caption={}]
package Strings;

public class ReverseString {
    
    public static void main(String[] args) {
        
        String str = "Hello";
        String rev = "";

        for(int i = str.length()-1; i>=0; i--){
            rev = rev + str.charAt(i);
        }
        System.out.println("Reversed String: "+ rev);

    }
}
\end{lstlisting}
\subsubsection*{Output}
% \vspace{-1em}
\begin{figure}[H]
    \centering
    \vspace{-1em}
    \includegraphics[width=1.0\textwidth]{Lab6/Images/sum.png}
    \caption{Output of Sum of Digits Program}
    \label{fig:sum_output}
\end{figure}

%-----------------------------------------------

% \refstepcounter{section}
\addlabcontentsline{Assignment \thesection: Linked List}{12-02-2026}{\thepage}
\section*{\centering Assignment \thesection: Linked List}
%-----------------------------------------------
% Task 1
% -----------------------------------------------
\subsection{Singly-Linked-List}
\begin{lstlisting}[style=pythonStyle, caption={}]
public class SinglyLinkedList {
    // Node class
    class Node{
        int data;
        Node next;
        Node(int data){
            this.data = data;
            this.next = null;
        }
    }

    Node head = null;
    // Insert at end
    public void insertEnd(int data){
        Node newNode = new Node(data);
        if(head == null){
            head = newNode;    
            return;
        }
        
        Node temp = head;
        while(temp.next != null){
            temp = temp.next;
        }
        temp.next = newNode;
    }

    // Insert at beginning
    public void insertBeginning(int data){
        Node newNode = new Node(data);
        newNode.next = head;
        head = newNode;
    }

    // Delete by value
    public void delete(int key){
        if (head == null) 
            return;
        if (head.data == key) {
            head = head.next;
            return;
        }

        Node temp = head;
        while (temp.next != null && temp.next.data != key) {
            temp = temp.next;
        }
        if(temp.next != null){
            temp.next = temp.next.next;
        }
    }

    // Traverse
    public void display(){
        Node temp = head;
        while(temp != null){
            System.out.print(temp.data + "->");
            temp = temp.next;
        }

        System.out.println("null");
    }

    // Driver code
    public static void main(String[] args) {
        SinglyLinkedList list = new SinglyLinkedList();
        list.insertEnd(10);
        list.insertEnd(20);
        list.insertEnd(30);
        System.out.println("---Insert End---");
        list.display();
        list.insertBeginning(1);
        list.insertBeginning(2);
        list.insertBeginning(3);
        list.insertBeginning(4);
        list.insertBeginning(5);
        System.out.println("---Insert Beginning---");
        list.display();

        list.delete(30);
        System.out.println("---Delete 30---");
        list.display();
    }


}

\end{lstlisting}
\subsubsection*{Output}
% \vspace{-1em}
\begin{figure}[H]
    \centering
    \vspace{-1em}
    \includegraphics[width=1.0\textwidth]{Lab7/Images/singly_ll.png}
    \caption{Singly Linked List Output}
    \label{fig:singly_linked_list_output}
\end{figure}
%-----------------------------------------------
% Task 2
% -----------------------------------------------
\subsection{Doubly-Linked-List}
\begin{lstlisting}[style=pythonStyle, caption={}]
// Doubly Linked List Program
public class DoublyLinkedList {
    // Node class
    class Node{
        int data;
        Node prev; // Previous 
        Node next; //Next Pointer
        Node(int data){
            this.data = data;
            this.prev = null;
            this.next = null;
        }
    }
    Node head = null;

    // Insert at beginning
    public void insertAtBeginning(int data){
        Node newNode = new Node(data);
        if(head == null){
            head = newNode;
            return;
        }
        newNode.next = head;
        head.prev = newNode;
        head = newNode;
    }

    // Insert at end
    public void insertAtEnd(int data){
        Node newNode = new Node(data);
        if(head == null){
            head = newNode;
            return;
        }
        Node temp = head;
        while (temp.next != null) {
            temp = temp.next;
        }
        temp.next = newNode;
        newNode.prev = temp;
    }

    // Insert at Position (1- Based Index)
    public void insertAtPosition(int data, int position){
        if (position <= 0) {
            System.out.println("Invalid Position");
            return;
        }
        if (position == 1) {
            insertAtBeginning(data);
            return;
        }
        Node newNode = new Node(data);
        Node temp = head;

        for(int i=1; temp != null && i<position-1; i++){
            temp = temp.next;
        }
        if (temp == null) {
            System.out.println("Position out of range");
            return;
        }
        newNode.next = temp.next;
        if (temp.next != null) {
            temp.next.prev = newNode;
        }
        temp.next = newNode;
        newNode.prev = temp;
    }

    // Delete from Beginning
    public void deleteFromBeginning(){
        if (head == null) {
            System.out.println("List is Empty");
            return;
        }
        head = head.next;
        if (head != null) {
            head.prev = null;
        }
    }

    // Delete from End
    public void deleteFromEnd(){
        if (head == null) {
            System.out.println("List is Empty");
            return;
        }
        if (head.next == null) {
            head = null;
            return;
        }
        Node temp = head;
        while (temp.next != null) {
            temp = temp.next;
        }
        temp.prev.next = null;
    }

    // Delete from Position
    public void deleteFromPosition(int position){
        if (head == null || position <= 0) {
            System.out.println("Invalid Operation");
            return;
        }
        if(position == 1){
            deleteFromBeginning();
            return;
        }
        Node temp = head;
        for(int i =1; temp != null && i < position; i++){
            temp = temp.next;            
        }
        if (temp == null) {
            System.out.println("Position out of Range.");
            return;
        }
        if (temp.next != null) {
            temp.next.prev = temp.prev;
        }
        if (temp.prev != null) {
            temp.prev.next = temp.next;
        }
    }

    // Search
    public void search(int key){
        Node temp = head;
        int position = 1;
        while (temp != null) {
            if (temp.data == key) {
                System.out.println("Element found at position: "+ position);
                return;
            }
            temp = temp.next;
            position++;
        }
        System.out.println("Element Not Found!");
    }

    // Display forward
    public void displayForward(){
        Node temp = head;
        System.out.print("Forward: ");
        while (temp != null) {
            System.out.print(temp.data + " ");
            temp = temp.next;
        }
        System.out.println();
    }
    // Display Backward
    public void displayBackward(){
        if (head == null) 
            return;
        Node temp = head;
        while (temp.next != null) {
            temp = temp.next;
        }    
        System.out.print("Backward: ");
        while (temp != null) {
            System.out.print(temp.data + " ");
            temp = temp.prev;
        }
        System.out.println();
    }

    // Driver code
    public static void main(String[] args) {
        DoublyLinkedList dll = new DoublyLinkedList();
        dll.insertAtBeginning(10);
        dll.insertAtBeginning(5);
        dll.insertAtBeginning(50);
        dll.insertAtBeginning(100);
        dll.insertAtEnd(20);
        dll.insertAtEnd(1000);
        dll.insertAtEnd(2000);
        dll.insertAtEnd(3000);
        dll.insertAtEnd(25);
        dll.insertAtPosition(15, 3);
        dll.insertAtPosition(500, 1);
        dll.displayForward();
        dll.displayBackward();
        dll.search(15);
        dll.deleteFromBeginning();
        dll.deleteFromEnd();
        dll.deleteFromPosition(2);
        dll.displayForward();
        dll.displayBackward();

    }
}

\end{lstlisting}
\subsubsection*{Output}
% \vspace{-1em}
\begin{figure}[H]
    \centering
    \vspace{-1em}
    \includegraphics[width=1.0\textwidth]{Lab7/Images/doubly_ll.png}
    \caption{Output of Doubly Linked List Program}
    \label{fig:doubly_linked_list_output}
\end{figure}

% \include{Lab8/Lab_8}
\clearpage
% \bibliographystyle{plainnat}
% \bibliography{refs/example.bib}
% \nocite{*}
% \printindex

\end{document}
