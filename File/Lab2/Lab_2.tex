\refstepcounter{section}
\addlabcontentsline{Assignment \thesection: Sorting Algorithms}{02-02-2026}{\thepage}
\section*{\centering Assignment \thesection: Sorting Algorithms}
%-----------------------------------------------
% Task 1
% -----------------------------------------------
\subsection{Write a program to show Bubble Sort}
\begin{lstlisting}[style=pythonStyle, caption={}]
// Sorting of array in Bubble Sort
public class BubbleSort{
    public static void main(String[] args){
        int arr[] = {5,1,4,2,8};
        for(int i=0; i<arr.length-1; i++){
            for(int j=0; j<arr.length-i-1; j++){
                if(arr[j]>arr[j+1]){
                    int temp = arr[j];
                    arr[j] = arr[j+1];
                    arr[j+1] = temp;
                }
            }
        }
        for(int num : arr){
            System.out.print(num + " ");
        }
    }
}
\end{lstlisting}
\subsubsection*{Output}
% \vspace{-1em}
\begin{figure}[H]
    \centering
    \vspace{-1em}
    \includegraphics[width=1.0\textwidth]{Lab2/Images/bubble_sort.png}
    \caption{Program to show Bubble Sort}
    \label{fig:bubble_sort_output}
\end{figure}
%-----------------------------------------------
% Task 2
% -----------------------------------------------
\subsection{Write a program to show Insertion Sort}
\begin{lstlisting}[style=pythonStyle, caption={}]
// Insertion Sort
public class InsertionSort {
    public static void main(String[] args) {
        int arr[] = {1,4,5,2,3,7,9,8,0,6};
        for(int i = 1;i<arr.length; i++){
            int key = arr[i];
            int j = i-1;
            while (j>=0 && arr[j]>key) {
                arr[j+1] = arr[j];
                j--;
            }
            arr[j+1] = key;
        }
        for(int num : arr){
            System.out.print(num + " ");
        }
    }    
}
\end{lstlisting}
\subsubsection*{Output}
% \vspace{-1em}
\begin{figure}[H]
    \centering
    \vspace{-1em}
    \includegraphics[width=1.0\textwidth]{Lab2/Images/insertion_sort.png}
    \caption{Program to show Insertion Sort}
    \label{fig:insertion_sort_output}
\end{figure}
%-----------------------------------------------    
% Task 3
% -----------------------------------------------
\subsection{Write a program to show Selection Sort}
\begin{lstlisting}[style=pythonStyle, caption={}]
    //Selection Sort 
public class SelectionSort {
    public static void main(String[] args) {
        int arr[] = {3,5,2,6,-1,0,4,7,9,8,-2};

        for(int i=0;i<arr.length-1;i++){
            int minIndex = i;
            for(int j = i+1; j<arr.length;j++){
                if(arr[j]<arr[minIndex]){
                    minIndex = j;

                }
            }
            int temp = arr[minIndex];
            arr[minIndex] = arr[i];
            arr[i] = temp;
        }
        for(int num : arr){
            System.out.print(num + " ");

        }
    }
    
}
\end{lstlisting}
\subsubsection*{Output}
% \vspace{-1em}
\begin{figure}[H]
    \centering
    \vspace{-1em}
    \includegraphics[width=1.0\textwidth]{Lab2/Images/selection_sort.png}
    \caption{Program to show Selection Sort}
    \label{fig:selection_sort_output}
\end{figure}
%-----------------------------------------------
% Task 4
% -----------------------------------------------
\subsection{Write a program to show Merge Sort}
\begin{lstlisting}[style=pythonStyle, caption={}]
    //Merge Sort : Divide and Conquer
import java.util.Arrays;

public class MergeSort {
   
    // Main merge sort function
    public static void mergeSort(int[] arr, int left, int right){
        if(left < right){
            int mid =  left + (right-left)/2;
            // Sort Left Half
            mergeSort(arr, left, mid);
            // Sort Right Half
            mergeSort(arr, mid+1, right);
            // Merge Both Halves
            merge(arr, left, mid, right); 
        }
    }

    // Merge two sorted subarrays
    public static void merge(int[] arr, int left, int mid, int right){
        int n1 = mid-left+1;
        int n2 = right-mid;
        int[] l = new int[n1];
        int[] r = new int[n2];

        // Copy data to temp arrays
        for(int i=0; i<n1; i++)
            l[i] = arr[left+i];
        for(int j=0; j<n2; j++)
            r[j] = arr[mid+1+j];

        int i = 0, j=0, k=left;

        // Merge temp arrays back into array/arr
        while (i<n1 && j<n2) {
            if (l[i]<=r[j]) {
                arr[k++] = l[i++];
            }
            else{
                arr[k++] = r[j++];
            }
        }

        // Copy remaining elements
        while (i<n1) {
            arr[k++] = l[i++];
        }
        while (j<n2) {
            arr[k++] = r[j++];
        }
    }

    // Driver Code
    public static void main(String[] args) {
        
        int[] arr = {38,27,43,3,9,82,10};
        System.out.println("Before Sorting: ");
        System.out.println(Arrays.toString(arr));

        mergeSort(arr, 0, arr.length-1);
        
        System.out.println("After Sorting: ");
        System.out.println(Arrays.toString(arr));

    }
}
\end{lstlisting}
\subsubsection*{Output}
% \vspace{-1em}
\begin{figure}[H]
    \centering
    \vspace{-1em}
    \includegraphics[width=1.0\textwidth]{Lab2/Images/merge_sort.png}
    \caption{Program to show Merge Sort}
    \label{fig:merge_sort_output}
\end{figure}
%-----------------------------------------------
% Task 5
% -----------------------------------------------
\subsection{Write a program to show Quick Sort}
\begin{lstlisting}[style=pythonStyle, caption={}]
    // Quick Sort : small -- pivot -- larger
import java.util.Arrays;

public class QuickSort {

    // Main quick sort function
    public static void quickSort(int[] arr, int low, int high){
        if (low<high) {
            int pivotIndex = partition(arr, low, high);
            
            // Sort elements before and after partition
            quickSort(arr, low, pivotIndex-1);
            quickSort(arr, pivotIndex+1, high);
        }
    }

    // Partition Function 
    public static int partition(int[] arr, int low, int high){
        int pivot = arr[high]; // Choose last element as pivot
        int i = low-1;
        for(int j=low; j<high; j++){
            if (arr[j]<pivot) {
                i++;

                // Swap arr[i] and arr[j]
                int temp = arr[i];
                arr[i] = arr[j];
                arr[j] = temp;
            }
        }

        // Place pivot at correct position
        int temp = arr[i+1];
        arr[i+1] = arr[high];
        arr[high] = temp;

        return i+1;
    }

    // Driver code
    public static void main(String[] args) {
        
        int[] arr = {34,2,56,4,89,65,1,44};
        System.out.println("Before Sorting: ");
        System.out.println(Arrays.toString(arr));

        quickSort(arr, 0, arr.length-1);
        System.out.println("After Sorting: ");
        System.out.println(Arrays.toString(arr));

    }
}
\end{lstlisting}
\subsubsection*{Output}
% \vspace{-1em}
\begin{figure}[H]
    \centering
    \vspace{-1em}
    \includegraphics[width=1.0\textwidth]{Lab2/Images/quick_sort.png}
    \caption{Program to show Quick Sort}
    \label{fig:quick_sort_output}
\end{figure}
%-----------------------------------------------
