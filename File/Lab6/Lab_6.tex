\refstepcounter{section}
\addlabcontentsline{Assignment \thesection: Recursion}{11-02-2026}{\thepage}
\section*{\centering Assignment \thesection: Recursion}
%-----------------------------------------------
% Task 1
% -----------------------------------------------
\subsection{Factorial of a Number using Recursion}
\begin{lstlisting}[style=pythonStyle, caption={}]
package Recursion;

import java.util.Scanner;

// Factorial Program --> eg: 4! = 4*3*2*1 = 24
public class Factorial {

    static int factorial(int n){
        if (n == 0) 
            return 1;
        return n * factorial(n-1);
    }

    //Driver Code
    public static void main(String[] args) {

        System.out.println("Enter any number: ");
        Scanner sc = new Scanner(System.in);
        int num = sc.nextInt();
        System.out.println("Factorial of "+ num + " is "+ factorial(num));
        sc.close();
    }
}

\end{lstlisting}
\subsubsection*{Output}
% \vspace{-1em}
\begin{figure}[H]
    \centering
    \vspace{-1em}
    \includegraphics[width=1.0\textwidth]{Lab6/Images/factorial.png}
    \caption{Output of Factorial Program}
    \label{fig:factorial_output}
\end{figure}
%-----------------------------------------------
% Task 2
% -----------------------------------------------
\subsection{Fibonacci Series using Recursion}
\begin{lstlisting}[style=pythonStyle, caption={}]
package Recursion;
import java.util.Scanner;

//Fibonacci Series --> 0,1,0+1=1,1+1=2,2+1=3...
public class Fibonacci {
    static int fibonacci(int n){
        if (n == 0)
            return 0;
        if (n == 1)
            return 1;
        return fibonacci(n-1) + fibonacci(n-2);
    }
    //Driver Code
    public static void main(String[] args) {

        Scanner sc = new Scanner(System.in);
        System.out.println("Enter Element: ");
        
        int terms = sc.nextInt();
        for(int i = 0; i<terms; i++){
            System.out.print(fibonacci(i) + " ");
            sc.close();
        } 
    }
}

\end{lstlisting}
\subsubsection*{Output}
% \vspace{-1em}
\begin{figure}[H]
    \centering
    \vspace{-1em}
    \includegraphics[width=1.0\textwidth]{Lab6/Images/fibonacci.png}
    \caption{Output of Fibonacci Program}
    \label{fig:fibonacci_output}
\end{figure}
%-----------------------------------------------    
% Task 3
% -----------------------------------------------
\subsection{Palindrome String using Recursion}
\begin{lstlisting}[style=pythonStyle, caption={}]
package Strings;
// Remove spaces from string
public class RemoveSpaces {
    public static void main(String[] args) {
        String str = "Every Expert Was Once A Beginner.";
        System.out.println(str.replace(" ", ""));
    }
}

\end{lstlisting}
\subsubsection*{Output}
% \vspace{-1em}
\begin{figure}[H]
    \centering
    \vspace{-1em}
    \includegraphics[width=1.0\textwidth]{Lab6/Images/palindrome.png}
    \caption{Output of Palindrome Program}
    \label{fig:palindrome_output}
\end{figure}
%-----------------------------------------------
% Task 4
% -----------------------------------------------
\subsection{Power of a number using Recursion}
\begin{lstlisting}[style=pythonStyle, caption={}]
package Strings;

public class ReverseString {
    
    public static void main(String[] args) {
        
        String str = "Hello";
        String rev = "";

        for(int i = str.length()-1; i>=0; i--){
            rev = rev + str.charAt(i);
        }
        System.out.println("Reversed String: "+ rev);

    }
}
\end{lstlisting}
\subsubsection*{Output}
% \vspace{-1em}
\begin{figure}[H]
    \centering
    \vspace{-1em}
    \includegraphics[width=1.0\textwidth]{Lab6/Images/power_of_number.png}
    \caption{Output of Power of Number Program}
    \label{fig:power_of_number_output}
\end{figure}

%-----------------------------------------------
% Task 5
% -----------------------------------------------
\subsection{Print numbers using Recursion}
\begin{lstlisting}[style=pythonStyle, caption={}]
package Strings;

public class ReverseString {
    
    public static void main(String[] args) {
        
        String str = "Hello";
        String rev = "";

        for(int i = str.length()-1; i>=0; i--){
            rev = rev + str.charAt(i);
        }
        System.out.println("Reversed String: "+ rev);

    }
}
\end{lstlisting}
\subsubsection*{Output}
% \vspace{-1em}
\begin{figure}[H]
    \centering
    \vspace{-1em}
    \includegraphics[width=1.0\textwidth]{Lab6/Images/numbers_print.png}
    \caption{Output of Print Numbers Program}
    \label{fig:numbers_print_output}
\end{figure}

%-----------------------------------------------
% Task 6
% -----------------------------------------------
\subsection{Recursion Factorial of a Number}
\begin{lstlisting}[style=pythonStyle, caption={}]
package Strings;

public class ReverseString {
    
    public static void main(String[] args) {
        
        String str = "Hello";
        String rev = "";

        for(int i = str.length()-1; i>=0; i--){
            rev = rev + str.charAt(i);
        }
        System.out.println("Reversed String: "+ rev);

    }
}
\end{lstlisting}
\subsubsection*{Output}
% \vspace{-1em}
\begin{figure}[H]
    \centering
    \vspace{-1em}
    \includegraphics[width=1.0\textwidth]{Lab6/Images/recurse_factorial.png}
    \caption{Output of Recursion Factorial Program}
    \label{fig:recurse_factorial_output}
\end{figure}

%-----------------------------------------------
% Task 7
% -----------------------------------------------
\subsection{Reverse Number using Recursion}
\begin{lstlisting}[style=pythonStyle, caption={}]
package Strings;

public class ReverseString {
    
    public static void main(String[] args) {
        
        String str = "Hello";
        String rev = "";

        for(int i = str.length()-1; i>=0; i--){
            rev = rev + str.charAt(i);
        }
        System.out.println("Reversed String: "+ rev);

    }
}
\end{lstlisting}
\subsubsection*{Output}
% \vspace{-1em}
\begin{figure}[H]
    \centering
    \vspace{-1em}
    \includegraphics[width=1.0\textwidth]{Lab6/Images/reverse_number.png}
    \caption{Output of Reverse Number Program}
    \label{fig:reverse_number_output}
\end{figure}

%-----------------------------------------------
% Task 8
% -----------------------------------------------
\subsection{String Recursion Menu}
\begin{lstlisting}[style=pythonStyle, caption={}]
package Strings;

public class ReverseString {
    
    public static void main(String[] args) {
        
        String str = "Hello";
        String rev = "";

        for(int i = str.length()-1; i>=0; i--){
            rev = rev + str.charAt(i);
        }
        System.out.println("Reversed String: "+ rev);

    }
}
\end{lstlisting}
\subsubsection*{Output}
% \vspace{-1em}
\begin{figure}[H]
    \centering
    % \vspace{-1em}
    \includegraphics[width=0.8\textwidth]{Lab6/Images/menu1.png}
    % \includegraphics[width=1.0\textwidth]{Lab6/Images/menu2.png}
    \caption{Output of String Recursion Menu Program}
    \label{fig:string_recursion_menu_output}
\end{figure}
\begin{figure}[H]
    \centering
    % \vspace{-1em}
    % \includegraphics[width=1.0\textwidth]{Lab6/Images/menu1.png}
    \includegraphics[width=0.8\textwidth]{Lab6/Images/menu2.png}
    \caption{Output of String Recursion Menu Program}
    \label{fig:string_recursion_menu_output_2}
\end{figure}

%-----------------------------------------------
% Task 9
% -----------------------------------------------
\subsection{Sum of digits using Recursion}
\begin{lstlisting}[style=pythonStyle, caption={}]
package Strings;

public class ReverseString {
    
    public static void main(String[] args) {
        
        String str = "Hello";
        String rev = "";

        for(int i = str.length()-1; i>=0; i--){
            rev = rev + str.charAt(i);
        }
        System.out.println("Reversed String: "+ rev);

    }
}
\end{lstlisting}
\subsubsection*{Output}
% \vspace{-1em}
\begin{figure}[H]
    \centering
    \vspace{-1em}
    \includegraphics[width=1.0\textwidth]{Lab6/Images/sum.png}
    \caption{Output of Sum of Digits Program}
    \label{fig:sum_output}
\end{figure}

%-----------------------------------------------
